\documentclass{CV_Chinnapareddy_2220570} 
\usepackage[left=0.75in,top=0.6in,right=0.75in,bottom=0.6in]{geometry} 
\newcommand{\tab}[1]{\hspace{.2667\textwidth}\rlap{#1}}
\newcommand{\itab}[1]{\hspace{0em}\rlap{#1}}
\name{Chinnapareddy Krishna Vamsi}
\address{(+91)8341192296 \\ kvamc.ch@gmail.com}

\begin{document}

%%%%%%%%%%%%%%%%%%%%%%%%%%%%%%% Education %%%%%%%%%%%%%%%%%%%%%%%%%%%%%%%
\begin{rSection}{Education}
{\bf National Institute of Technology Karnataka, Surathkal} \hfill {\em Aug 2022 - Jul 2024(Expected)} 
\\ Masters in Computer Science.

{\bf V.R Siddhartha Engineering College} \hfill {\em Aug 2015 - May 2019} 
\\ Bachelors in Information Technology. \hfill {\em Overall CGPA: 8.3/10}
\end{rSection}

%%%%%%%%%%%%%%%%%%%%%%%%%%%%%%% Career Objective %%%%%%%%%%%%%%%%%%%%%%%%%%%%%%%
\begin{rSection}{Career Objective}
 To work for an organization which provides me the opportunity to improve my skills and knowledge to grow along with the organization objective.
\end{rSection}

%%%%%%%%%%%%%%%%%%%%%%%%%%%%%%% Work Experience %%%%%%%%%%%%%%%%%%%%%%%%%%%%%%%
\begin{rSection}{Work Experience}
\begin{rSubsection}{Mphasis, India}{Jul 2019 - Oct 2021}{Software Engineer}{}
 \item Good knowledge of software QA methodologies, automation tools and processes.
 \item Developed test scripts for automating the regression process.
 \item Tested cross platform web apps using postman API.
 \end{rSubsection}
\end{rSection}

%%%%%%%%%%%%%%%%%%%%%%%%%%%%%%% Technical Skills %%%%%%%%%%%%%%%%%%%%%%%%%%%%%%%
\begin{rSection}{Technical Skills}

\begin{tabular}{ @{} >{\bfseries}l @{\hspace{6ex}} l }
Languages \ & C, C++, Python, Java,  \\
Technologies & HTML5, CSS, Latex, Javascript, PHP \\
Tools & TexMaker, Sublime, Jira, MySql, Postman, Weka, Rapid Miner, Jupyter Notebook \\
\end{tabular}

\end{rSection}

%%%%%%%%%%%%%%%%%%%%%%%%%%%%%%% Projects %%%%%%%%%%%%%%%%%%%%%%%%%%%%%%%
\begin{rSection}{Projects}
{\bf Recommendation Engine}
\\A novel fashion evaluation method on the basis of the content based filtering in this
paper. The features are extracted from data set according to the characters’ facial feature
method using WordtoVec algorithm. The experimental results show that most similar products that can be accurately described based on the consumers’ interest and the approach has higher application value for the clothing recommendation system.

{\bf Self-Help Groups System}
\\A self-help group (SHG) is a financial intermediary committee usually
composed of 10–20 local women. Self-help groups are started by -governmental
organizations (GO) that generally have broad anti-poverty agendas. Members also make
small regular savings contributions over a few months until there is enough money in the
group to begin lending. To automate this system here we are developing a mobile Android
application which gives all reports related to SHG groups in Andhra Pradesh.

\end{rSection}

%%%%%%%%%%%%%%%%%%%%%%%%%%%%%%% Achievements %%%%%%%%%%%%%%%%%%%%%%%%%%%%%%%
\begin{achSection}{Achievements} 
\item Cracked GATE exam with 98.3 percentile.
\item Secured 2nd prize for developing Wireless Charger among 300 projects on Engineer's day 2k16 organized by V.R Siddhartha Engineering College.
\item Bagged 2nd prize for designing Quadcopter on Engineer's Day 2k17 organized by V.R Siddhartha Engineering College.
\end{achSection}


%%%%%%%%%%%%%%%%%%%%%%%%%%%%%%% Co-Curricular Activities %%%%%%%%%%%%%%%%%%%%%%%%%%%%%%%
\begin{achSection}{Co-Curricular Activities} 

\item Event Coordinator and Chief Organizer for the technical event "Coding Contest"
held at 'AFOSEC - Annual Festival Of Siddhartha Engineering College'.

\item Department Coordinator and Chief Organizer for the technical and cultural club
‘SUMMIT’ at VR Siddhartha Engineering College.

\item Accumulated knowledge in the arena “Fundamentals of android programming” by
participating in a three day hands on workshop organized by "APSSDC,
Vijayawada" in association with Google at VR Siddhartha Engineering College.

\item Accumulated knowledge in the arena "Block Chain" which is the cutting edge
technology in the background of cryptocurrency held at VR Siddhartha
Engineering College.

\end{achSection}


%%%%%%%%%%%%%%%%%%%%%%%%%%%%%%% Extra-Curricular Activities %%%%%%%%%%%%%%%%%%%%%%%%%%%%%%%
\begin{achSection}{Extra-Curricular Activities} 

\item Participated in a event on “Shaastra Programming Contest” held on 4th Janat SHAASTRA
2018, the National Technical festival of IIT Madras campus.

\item Participated in a two day workshop on “Cloud Computing” held on 14th and 16th
October at ATMOS 2016, the National Techno-Management festival of BITS Pilani
Hyderabad campus.

\item Participated in a oneday workshop on “Web Development” held on 5th Janat
SHAASTRA 2018, the National Technical festival of IIT Madras campus.

\end{achSection}

%%%%%%%%%%%%%%%%%%%%%%%%%%%%%%% Certifications %%%%%%%%%%%%%%%%%%%%%%%%%%%%%%%
\begin{achSection}{Certifications} 

\item Certified in "Data Structures and Algorithms using python" by NPTEL.

\item Certified in "Introduction to Data Science" by NPTEL.

\item Certified in "IBM bluemix" by IBM Cloud.

\end{achSection}

\end{document}
